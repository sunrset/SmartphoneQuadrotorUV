\chapter*{Abstract} \label{abstract}
\addcontentsline{toc}{chapter}{Abstract}
\setcounter{page}{5} 
Aerial robotics is a field in constant growth, both within the scientific community and in commercial applications. There are more and more existing development platforms, with which an unmanned aerial vehicle can be built without making large economic investments. On the other hand, smartphones have become devices with great computing and sensing capacity, allowing various scientific and research projects to be developed using these devices. It is then possible to think about integrating a smartphone as a processing and sensing device within an unmanned aerial system.
\\\\
This work aims to develop and implement control and estimation strategies which can be executed in a smartphone in order to control the flight dynamics of a rotary-wing unmanned aerial vehicle, specifically a quadrotor. Initially, the concept of unmanned aerial vehicle, its characteristics and classifications are introduced. Subsequently, the mathematical model of a quadrotor dynamics is obtained based on the Newton-Euler and Euler-Lagrange approaches. The components of the smartphone-based quadrotor are detailed and their specific parameters are experimentally identified. Two optimal control strategies are designed using the linearized model of the quadrotor. These control strategies are the linear quadratic regulator with integral feedback (LQI) and the $H_\infty$ controller. In addition, a Kalman filter is designed to estimate the non-measurable states of the system. After verifying using simulations, that the designed controllers are functional within the quadrotor, the flight controller is implemented in Android. This flight controller includes the control, state estimation and communication algorithms. Additionally, a desktop application is implemented to monitor and configure the quadrotor during its flight. Finally, the quadrotor is subjected to multiple tests in its flight modes, using both implemented controllers. The results suggest that the $H_\infty$ controller performs better than the LQI controller in the developed smartphone-based quadrotor prototype.
\\\\
\keywords{Quadrotor, Smartphone, Optimal control, $H_\infty$ controller, Linear quadratic controller}

\newpage
\thispagestyle{empty}
\mbox{}

\chapter*{Resumen} \label{resumen}
\addcontentsline{toc}{chapter}{Resumen}
%\pagenumbering{roman}
%\setcounter{page}{7} 
La robótica aérea es un campo en constante crecimiento, tanto dentro de la comunidad científica, como en aplicaciones comerciales. Cada vez son más las plataformas de desarrollo existentes, con las cuales puede construirse un vehículo aéreo no tripulado sin realizar grandes inversiones económicas. Por otra parte, los teléfonos inteligentes se han convertido en dispositivos con gran capacidad computacional y de sensado, permitiendo que se desarrollen diversos proyectos científicos y de investigación utilizando estos dispositivos. Es posible pensar en integrar un smartphone como dispositivo de procesamiento y sensado dentro de un sistema aéreo no tripulado. 
\\\\
Este trabajo tiene como objetivo desarrollar e implementar estrategias de control y estimación de estados que puedan ser ejecutadas en un teléfono inteligente con el fin de controlar las dinámicas de vuelo de un vehículo aéreo no tripulado de ala rotatoria, específicamente un quadrotor. Inicialmente, se introduce el concepto de vehículo aéreo no tripulado, sus características y clasificaciones. Posteriormente, se obtiene el modelo matemático de las dinámicas de un quadrotor basandose en las aproximaciones de Newton-Euler y Euler-Lagrange. Los componentes del quadrotor basado en smartphone son detallados y sus parámetros específicos son identificados experimentalmente. Dos estrategías de control óptimo son diseñadas a partir del modelo linealizado del quadrotor. Estas estrategías de control son el regulador cuadrático lineal con realimentación integral (LQI) y el controlador $H_\infty$. Además, se diseña un filtro de Kalman para estimar los estados no medibles del sistema. Tras verificar por medio de simulaciones, que los controladores diseñados son funcionales dentro del quadrotor, se implementa en Android el controlador de vuelo. Este controlador de vuelo incluye los algoritmos de control, de estimación de estados y de comunicación. Adicionalmente, se implementa una aplicación de escritorio para el monitoreo y configuración del quadrotor durante el vuelo. Finalmente, el quadrotor es sometido a múltiples pruebas en sus modos de vuelo, utilizando ambos controladores implementados. Los resultados sugieren que el controlador $H_\infty$ tiene un mejor desempeño que el controlador LQI en el prototipo desarrollado de quadrotor basado en smartphone.
\\\\
\palabrasclave{Quadrotor, Smartphone, Control óptimo, Controlador $H_\infty$, Controlador lineal cuadrático}
