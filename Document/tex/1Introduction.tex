\chapter{Introduction} \label{ch:introduction}

\section{Quadrotors}

\begin{figure}[h]
\begin{center}
\includegraphics[width=8.6cm]{lji500x4.jpg}    
\caption[LJI 500-X4 carbon fiber frame]{LJI 500-X4 carbon fiber frame\protect\footnotemark} 
\label{fig:quadframe}
\end{center}
\end{figure}
\footnotetext{LJI 500-X4 frame image taken from \url{https://goo.gl/hHfHQR}}



\section{Motivation}
Quadrotor control is a difficult and interesting problem. A quadrotor has six degrees of freedom (three translational and three rotational) and four independent inputs (forces applied by the motors). As established by \cite{Liu2015} and \cite{Lopez2015}, quadrotor dynamics are affected by nonlinearity, parameters perturbations, uncertainties and disturbances: this include unknown and variable payloads, aerodynamical parameters of the system, wind changes, and sensors inaccuracies. Numerous studies have been developed in designing optimal and robust controllers that allow unmanned aircraft systems (UAS) to fly and accomplish missions rejecting disturbances and being robust to parameter uncertainties as seen in \cite{Jung2014, Kohno2014, Shang2016, Salazar2014}.\\\\
Although there are embedded systems with high computational capacity that can serve as controllers of a quadrotor, smartphones are available, easily accessible for people and also have a large computational capacity, hence multiple instrumentation and communication elements integrated in the same device. The use of smartphones also facilitates distribution and installation of updates of the control application as it is a commonly known device and has application distribution platforms. Some attempts to joint smartphones and aerial robots have been made: in \cite{Pearce2014a}, a smartphone was used as a mission planner for a quadrotor and in \cite{ALEMARK2014a}, a smartphone was used as flight controller using its sensors and power to stabilize the quadrotor and control its altitude. A smartphone has been used as a flight controller and processing system for image-based positioning in a quadrotor, as shown in \cite{Loianno2015}.
\\\\
This research project aims to design and implement algorithms that will be executed in a smartphone to estimate and control quadrotor dynamics by the Research Group in Industrial Control. This project confronts several challenges such as using a smartphone as a hardware development platform, trying to use a non real-time operating system for real-time applications, designing optimal and high order controllers using Java or C++, and executing that controllers in a smartphone.
\\\\
This paper presents the design of an optimal and a robust controller, and a state estimator based on a Kalman filter for a smartphone-based quadrotor. The non-linear and linerized model of the quadrotor is presented, the controller that allows the quadrotor to follow a trajectory reference using two approachs is designed. The two approaches are the LQG and $H_\infty$ controllers. The quadrotor flight dynamics estimation strategies using sensor fusion algorithms are described. 
\\\\
In recent years, the interest in aerial robotics research has increased substantially. This is because this type of robotics offers several potential new services such as search and rescue, observation, mapping, inspection, etc. On the other hand, smartphones have become essential devices for humans and easily acquirable development tools. The interaction between these two technologies allow the development of low cost quadcopters based on an everyday item such as the smartphones, facilitating the distribution of the quadcopter control software and its implementation by other researchers.
%~\ref{fig:quads500} for an example.
\begin{figure}[h]
\begin{center}
\includegraphics[width=8.6cm]{quadBasic1}    
\caption{Assembled quadcopter used in this research with the on-board smartphone on the top center of it.} 
\label{fig:quads500}
\end{center}
\end{figure}
\\\\
In this paper, the implementation of a quadcopter with a smartphone acting as its flight controller while using exclusively the sensors and processor in the smartphone, is shown. The controller keeps the attitude of the quadcopter stabilized while making the quadcopter to hover at an altitude reference. It is presented the detailed composition of the test platform (quadcopter) used, integrating it with its dynamic model in addition to the quadcopter altitude and attitude estimation strategies using sensor fusion algorithms.

\section{Research Problem}
The existing research challenges include how to develop and implement efficient control algorithms for smartphones using the Android operating system, and assess, adapt and develop the appropriate communication, sensing and performance technologies with smartphones in the execution of missions using quadrotors.
\\\\
Then, the question to be answered is: how to develop control strategies in a smartphone in order to control the flight dynamics of a quadrotor so that it can develop flight missions while the instrumentation and computing capacity of the smartphone is used?
\section{Objectives}
In order to find a solution for the research problem, the following general and specific objectives are proposed:
\subsubsection{General Objective}
Design and implement algorithms for control and estimation of flight dynamics executed in a smartphone for the quadrotor of the Industrial Control Research Group.
\subsubsection{Specific Objectives}
\begin{enumerate}
\item Conduct a study and analysis of the state of the art related to the control and estimation of states of quadrotors.
\item Integrate the existing quadrotor with a smart phone that contains the appropriate sensors for the control and estimation of states.
\item Obtain a dynamic model of the quadrotor.
\item Design and implement the algorithms for control and estimation of states for the quadrotor.
\item Integrate the experimentation platform with the control and estimation algorithms in the smartphone.
\item Evaluate the performance of the control strategies.%????
\end{enumerate}

\section{Literature Review}

\subsection{Control Strategies in Quadrotors}


\subsection{State Estimation in Quadrotors}


\subsection{Quadrotor Flight Modes}
%\url{http://ardupilot.org/copter/docs/flight-modes.html}
In quadrotors, the on-board flight controllers keep some of the quadrotors $DoF$ in a desired value autonomously in order to allow pilots to perform tasks during a flight. This controllers have different modes that can control from $3$ to $6$ $DoF$ depending on the will of the pilot. Flight modes commonly found in commercial flight controllers may be as basic to only control its attitude or as complex to let the quadrotor follow a complex trajectory with multiple waypoints \cite{Ardupilot2016}.
\\\\
The main flight modes, widely used in commercial flight controllers, are described bellow according to the number of controlled $DoF$, in ascending order.

\begin{itemize}
\item \textbf{Stabilize Mode}\\\\
This mode allows the pilot to fly the quadrotor manually while the flight controller self-levels the quadrotor attitude and regulate its current heading. Thus, the stabilize mode attempts to control three $DoF$ of the quadrotor.\\\\
The attitude references can be set or changed by the pilot using the remote control, but their default value is $0\ rad$. On the other hand, the quadrotor heading is simply set to be regulated in its current state, enabling its rate using the remote control.
\\\\
As the stabilize mode do not take into account the control of the quadrotor position, the pilot needs to regularly change the attitude references manually to keep the quadrotor in a desired position, as it is affected by wind disturbances. Also, the pilot needs to regularly adjust the quadrotor thrust, so that a desired altitude is maintained.

\item \textbf{Altitude Hold Mode}\\\\
The altitude hold mode adds automatic elevation control to the Stabilize mode. This way, in addition to controlling the attitude, the quadrotor thrust is set by the flight controller in order to maintain the quadrotor in a desired altitude, getting four controlled $DoF$.
\\\\
In this mode, the pilot can remotely control the rate of change of the elevation (with a default value of $0\ m/s$), as well as the attitude references.

\item \textbf{Loiter Mode}\\\\
Loiter mode automatically attempts to regulate the six $DoF$ of the quadrotor in order to maintain a desired position, heading and altitude during a flight. Here, the quadrotor attitude is self-leveled, while the position and altitude reference can be modified by the pilot using the remote controller. The position and altitude references are initialized using the current quadrotor position and altitude when this mode is set. This is a GNSS-Dependent flight mode.

\item \textbf{Auto Mode}\\\\
The Auto mode attempts to make a quadrotor follow automatically a pre-programmed path connecting multiple position and heading waypoints, while receiving . This mode use the same controller as the Loiter mode, but its references are set automatically following the waypoints list. As in Auto mode the quadrotor must follow geo-located waypoints, this is a GNSS-Dependent flight mode.

\item \textbf{Return-To-Launch Mode}\\\\
During a flight mission, the home location is set as the position and altitude where the quadrotor took off. The Return-To-Launch mode is used in case of emergency or when a the last waypoint is reached within a flight mission. This mode is equivalent to the Auto mode, but only has two waypoints. The first waypoint consists in the position of the home location with a previously set altitude ($RTL$ altitude) greater than the home altitude. When this waypoint is reached, the home location is set as the following waypoint so that the quadrotor starts its landing keeping the position controlled.

\end{itemize}


\subsection{Smartphones in Control Systems}
Current smartphone processors are able to perform complex calculations such as those required in the implementation of real time control strategies. There are many ongoing research related to the possibility of using smartphones to implement control strategies, such as \cite{Drumea2013a}, as configuration and monitoring interfaces in control systems as seen in \cite{Lin2014a,Truong2012a}, and as a tool in both education and design of control strategies seen in \cite{Aristizabal2014a,WuWu2013a}. Following this trend, in the Universidad del Valle, it was developed a smartphone-based platform for monitoring, control and communication in portable laboratories, where a controller for a pendulum based in the Lego Mindstorms EV3 platform was implemented \cite {GarciaTellez2015}.
\\\\
\cite{Gunawan2014}
\cite{Stefka2016}
\cite{Qgurlg2015}
\cite{Luo2014}
\cite{Lu2017}
\cite{Chen2011}
\cite{Tetzlaff2013}
\cite{Geissbuhler2015}
\cite{Oros2013a}
\cite{DeABarbosa2015}

\subsection{Smartphone-based Quadrotors}
\cite{Alsharif2017a}
\cite{Alsharif2016}
\cite{Alsharif2017}
\cite{Loianno2017}
\cite{Isuru2017}

In the University of Pennsylvania, in \cite{Loianno2015}, was developed a quadcopter using a last generation smartphone as a flight controller and an additional processing system for image-based positioning. The state estimation algorithms, control and planning were firstly implemented in a ODROID-XU board with additional sensors, but then, in \cite{Loianno2015a}, this algorithms were ported to the Qualcomm processor in the phone due to the Qualcomm colaboration in that project.
\\\\
Current research focuses on the development of aerial robots potentiated by the use of smartphones, as seen in \cite{Pearce2014a, ALEMARK2014a, Aldrovandi2015, Bryant2015}. In the last years, computing capacity and sensor technology in smartphones has decreased in price but increased in performance. Smartphones have become an inexpensive tool capable of commanding an UAV. The challenge then, is to use smartphones as quadcopter flight controllers for autonomous flights following specific missions, taking advantage of the fact that the phones today are very powerful computers that include elements of sensing, processing and signal communication.
\\\\

\subsubsection{Smartphone-based Quadrotor Limitations}
The idea of using a smartphone as flight controller in an UAV opens the possibility of a quick and inexpensive development \cite{Aldrovandi2015}. A smartphone offers other advantages compared with off-the-shelf flight controllers, for instance its powerful quad, hexa or octa-core processors and communications interfaces. However, smartphones and the Android operating system have some limitations that set challenges when implementing a control system in it.\\\\
Android is not a real-time operating system and thus can not assure execution of algorithms, like estimation and control, with a constant sample time. Furthermore, the sensors embedded in commercial smartphones are made for applications that do not have high requirements of accuracy nor precision and therefore may not be appropriate for sensing quadrotor dynamics. Nonetheless, as explained by \cite{Bryant2015}, due to its computing capabilities, smartphones can overcome this limitations while using a temporized thread to execute the control system algorithms and implementing a sensor fusion technique to improve the states estimation reliability. This thread must be executed with a lower sample time compared to the one of the sensors embedded in the smartphone. This will ensure that the execution is not delayed by the sensors acquisition process.

\section{Outline}
This thesis is organized as follows.\\\\
In Chapter \ref{ch:introduction}, 

\ref{ch:model}
\ref{ch:prototype}
\ref{ch:controlandestimation}
\ref{ch:implementation}
\ref{ch:conclusions}

In this chapter an introduction is given to this project and the quadrotor platform. Chapter 2
contains a overview of other main quadrotor platforms.
In the next part the dynamic model of the quadrotor is described, which is used to test
the indoor navigation and control in simulation. Because the quadrotor is open-loop unstable system,
a controller is required to do flight tests in simulation. This controller is designed in chapter 4.
In part III first the selected sensors that are used in the sensor integration are modelled.
Chapter 6 explains the state estimation method used to integrate the sensors of the IMU and the
IR sensors and discusses the results of this state estimation in simulation.
To perform flight tests with actual sensors, a quadrotor platform is designed. Part IV
starts with the development of this platform in chapter 7. Because without any control feedback
the quadrotor cannot fly, onboard filtering and control is required, which is described in chapter
8. Next in chapter 9 the real data from the quadrotor UAV is used to test the state estimation
method developed.
Finally, a conclusions and discussion of the work and recommendations for future work are
given.
