\setlength{\headheight}{15pt}
\pagestyle{fancy}
\fancyhead{}
\fancyfoot{}

\fancyhead[LE,RO]{\thepage}
\fancyhead[RE]{\slshape \leftmark}
\fancyhead[LO]{\slshape \rightmark}
\renewcommand{\headrulewidth}{0.4pt}
\renewcommand{\chaptermark}[1]{\markboth{\thechapter.\ #1}{}} % \MakeUppercase could be used
\renewcommand{\sectionmark}[1]{\markright{\thesection.\ #1}{}}

%\renewcommand{\algorithmicrequire}{\textbf{Input:}}
%\renewcommand{\algorithmicensure}{\textbf{Output:}}


%%%% OSCAR %%%
%\newcommand{\B}[1]{\mbox{\bm{$#1$}}}
%\newcommand{\mathbbm}[1]{\mbox{\bm{$#1$}}}
%\newcommand{\sPI}{\mbox{\bm{$\pi$}}}
%\newcommand{\rangeaxb}[3]{\ensuremath{#1 \leq #2 \leq #3}} 	% #1 <= #2 <= #3
%\newcommand{\sgn}{\mbox{\text{sgn}}} 						% sign operator
%\newcommand{\dH}{\ensuremath{\text{d}_\text{H}}} 			% Hamming distance operator
%\newcommand{\Pb}{\mbox{\text{P}}} 							% Probability operator 
%\newcommand{\Ex}{\mbox{\text{E}}} 							% Expectation operator
%\newcommand{\iround}{u} 									% Inner round index
%\newcommand{\sv}{\varsigma} 								% state vector probability 
%\newcommand{\BitP}{\mbox{$L_P$}} 							% length of the Package (bit-packaging mode)
%%\newcommand{\eqnref}[1]{Eq.~(\ref{#1})} 					% Ref. one equation
%%\newcommand{\eqsref}[2]{Eqs.~(\ref{#1}),(\ref{#2})} 		% Ref. two equations
%\newcommand{\figref}[1]{Fig.~\ref{#1}} 						% Ref. figures
%\newcommand{\Sref}[1]{Section~\ref{#1}}						% Ref. sections
%\newcommand{\sref}[1]{Section~\ref{#1}} 					% Ref. sections
%\newcommand{\pr}[2]{\ensuremath{p_{{#1},{\rm #2}}}} 		% Probability of #2 with parameter #1
%\newcommand{\SL}{0}   										% First index of s
%\newcommand{\SH}{G-1} 										% Last index of s

\newcommand{\emptypage}{\newpage{\pagestyle{empty}\cleardoublepage}} % Insert an empty page
%\includeonly{intro}

\usepackage{eqparbox,array}
%\renewcommand{\algorithmiccomment}[1]{\hfill\eqparbox{COMMENT}{// #1}}
\usepackage{algorithmic,eqparbox,array}
\renewcommand\algorithmiccomment[1]{%
  \hfill//\ \eqparbox{COMMENT}{#1}%
}
\newcommand\LONGCOMMENT[1]{%
  \hfill//\ \begin{minipage}[t]{\eqboxwidth{COMMENT}}#1\strut\end{minipage}%
}

% Code for creating empty pages
% No headers on empty pages before new chapter
\makeatletter
\def\cleardoublepage{\clearpage\if@twoside \ifodd\c@page\else
    \hbox{}
    \thispagestyle{empty}
    \newpage
    \if@twocolumn\hbox{}\newpage\fi\fi\fi}
\makeatother \clearpage{\pagestyle{empty}\cleardoublepage}

%\usepackage[colorlinks=true,linkcolor=black,urlcolor=black,citecolor=black,breaklinks]{hyperref}
\usepackage[colorlinks=true,linkcolor=blue,urlcolor=blue,citecolor=blue,breaklinks]{hyperref}
\usepackage[all]{hypcap}



%%% Esteban %%%%%%%%
\newtheorem{theorem}{Theorem}
\newtheorem{proposition}{Proposition}
\newtheorem{lemma}{Lemma}
\newtheorem{corollary}{Corollary}
\newtheorem{example}{Example}
\newtheorem{definition}{Definition}
\newtheorem{remark}{Remark}
\newtheorem{problem}{Problem}
\newtheorem{assumption}{Assumption}
%\begin{theorem}[Pierce’s Theorem]\label{T:set2}
%In category {\bf Set}, the monomorphisms are just the injective functions (the functions $f$
%such that $f(x)=f(y)$ implies $x=y.)$
%\label{T:set2}
%\end{theorem}
%
%\begin{proof}
%This is a proof that is ended by the standard q.e.d. symbol.
%\end{proof}
%
%\begin{proof}
%This is a proof that is ended with an equation and not the standard q.e.d. symbol.
%$$
%A=B.
%$$
%\renewcommand{\qedsymbol}{}
%\end{proof}
%
%\begin{proof}[Uniqueness]
%This is the proof.
%\end{proof}
%
%\begin{proof}[Proof of Theorem~\ref{T:set2}]
%State the proof.
%\end{proof}
%
%\begin{definition}[Definition]
%Here is a definition.
%\end{definition}
%
%\begin{lemma}[Lemma]
%Here is a lemma.
%\end{lemma}
%
%\begin{corollary}[Corollary to Theorem \ref{T:set2}]
%A corollary for theorem \ref{T:set2} is stated here, where 
%$C={A_3}^2$+$B$.
%\end{corollary}
%


%% TITLE %%
%\newcommand\AlCentroPagina[1]{%
%\AddToShipoutPicture*{\AtPageCenter{%
%\makebox(0,0){\includegraphics %
%[width =0.9\ paperwidth]{#1}}}}}


%% ABSTRACT %%

\newenvironment{abstract}%
{\cleardoublepage\null\vfill\begin{center}%
\bfseries\abstractname\end{center}}%
{\vfill\null}

%\newenvironment{abstract}%
%    {\cleardoublepage\thispagestyle{empty}\null\vfill\begin{center}%
%    \bfseries\abstractname\end{center}}%
%    {\vfill\null}



%% ACKNOWLEDGEMENTS %%
\newenvironment{acknowledgements}%
{\cleardoublepage\null\vfill\begin{center}%
\bfseries Acknowledgements\end{center}}%
{\vfill\null}


%\newenvironment{acknowledgements}%
%    {\cleardoublepage\thispagestyle{empty}\null\vfill\begin{center}%
%    \bfseries Acknowledgements\end{center}}%
%    {\vfill\null}




%% NOMENCLATURE %%

\renewcommand*\nomname{Nomenclature}
\setlength\nomlabelwidth{.25\linewidth} 
\setlength\nomitemsep{0.2\parsep} 
\newcommand\nomunit[1]{\def\nomentryend{\hfill#1}} 

\renewcommand\nomgroup[1]{% 
  \def\makelabel##1{##1}% 
  \bigskip 
  \ifx#1P\relax 
    \item[\textbf{\Large Parameters}]% 
  \fi 
  \ifx#1S\relax 
    \item[\textbf{\Large Symbols}]% 
  \fi 
  \ifx#1A\relax 
    \item[\textbf{\Large Abbreviations}]%     
  \fi 
  \medskip 
  \let\makelabel\nomlabel
  \vspace{10mm} 
} 



